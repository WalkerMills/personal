%%%%%%%%%%%%%%%%%%%%%%%%%%%%%%%%%%%%%%%%%%%%%%
% Mills Resume/CV
%
% Modification of Friggeri Resume/CV for software engineering
%
% Author:
% Walker Mills (walker.mills7@gmail.com)
%
% Based on:
%
%    %%%%%%%%%%%%%%%%%%%%%%%%%%%%%%%%%%%%%%%%%
%    % Friggeri Resume/CV
%    % XeLaTeX Template
%    % Version 1.0 (5/5/13)
%    %
%    % This template has been downloaded from:
%    % http://www.LaTeXTemplates.com
%    %
%    % Original author:
%    % Adrien Friggeri (adrien@friggeri.net)
%    % https://github.com/afriggeri/CV
%    %
%    % License:
%    % CC BY-NC-SA 3.0 (http://creativecommons.org/licenses/by-nc-sa/3.0/)
%    %
%    % Important notes:
%    % This template needs to be compiled with XeLaTeX and the bibliography, if used,
%    % needs to be compiled with biber rather than bibtex.
%    %
%    %%%%%%%%%%%%%%%%%%%%%%%%%%%%%%%%%%%%%%%%%
%
%%%%%%%%%%%%%%%%%%%%%%%%%%%%%%%%%%%%%%%%%%%%%%

\documentclass[]{mills-cv} % Add 'print' as an option into the square bracket to remove colors from this template for printing


\begin{document}

\header{walker}{mills}{software engineer} % Your name and current job title/field

%----------------------------------------------------------------------------------------
%	SIDEBAR SECTION
%----------------------------------------------------------------------------------------

\begin{aside} % In the aside, each new line forces a line break
\section{contact}
<<<<<<< HEAD
\small walker.mills7@gmail.com
(209) 450-6531
MSC 767
Pasadena, CA 91126
%bit.ly/WalkerMills
\section{languages}
English (native)
Spanish (fluent)
\section{programming}
\textbf{advanced:}
C, C++, Haskell,
Python, Cython, 
Linux/Unix, SQL,
Bash/Zsh
\textbf{proficient:}
Java, Pyrex
Erlang, IA32
Lisp, Lua
\section{technologies}
Django, Apache,
Thrift, NuoDB,
Protocol Buffers,
PostgreSQL, MySQL
Android, Chromium
\section{tools}
gcc/g++, make
gdb, git, gerrit
adb, mosh
=======
MSC 767
Pasadena, CA 91126
(209) 450-6531
\small walker.mills7@gmail.com
bit.ly/WalkerMills
\section{languages}
spanish (fluent)
\section{programming}
\textbf{advanced:}
C, C++, MySQL,
(C)Python, Cython
Haskell, Linux/Unix
Bash/Zsh
\textbf{proficient:}
PostgreSQL,
Pyrex, Java
NuoDB, Lisp
Erlang, IA32
\section{technologies}
Django, Thrift,
Protocol Buffers,
NewSQL, NoSQL
Android, Sage
Chromium
\section{tools}
GCC/G++, Make
SWIG, GDB, Git
Gerrit, ADB
SSH/Mosh
>>>>>>> 54d1cc3b582b224540ca2287790f91484845008b
\end{aside}

%----------------------------------------------------------------------------------------
%	EDUCATION SECTION
%----------------------------------------------------------------------------------------

\section{education}

\begin{entrylist}
%------------------------------------------------
\entry
{2012--2016}
{Bachelor of Computer Science \normalfont -- California Institute of Technology}{}{}
%------------------------------------------------
\end{entrylist}

%----------------------------------------------------------------------------------------
%	WORK EXPERIENCE SECTION
%----------------------------------------------------------------------------------------

\section{experience}

\begin{entrylist}
%------------------------------------------------
\entry
<<<<<<< HEAD
{Jul-Sept 2013}
{Amazon Lab126}
{Sunnyvale, California}
{\emph{Software Development Engineer Intern} \\
I worked on video streaming performance for Amazon WebView, a fork of Chromium, and assisted with the Mayday project for Amazon's Kindle Fire HDX devices. (NDA)
\begin{itemize}
\item I learned how to effectively use version control and issue tracking software in a professional environment.
\item I studied Chromium's source and traced program execution on production devices to document and optimize Chromium's multimedia internals
=======
{Summer 2013}
{Amazon Lab126}
{Sunnyvale, California}
{\emph{Software Development Engineer Intern} \\
Improved video streaming performance for Amazon WebView, a fork of Chromium, and worked on related video streaming projects for Amazon's Kindle Fire HDX devices. (NDA) \\
Detailed achievements:
\begin{itemize}
\item Learned how to incorporate version control into my daily work flow.
\item Taught my co-workers how Chromium's multimedia internals work.
\begin{itemize}
\item Examined the Chromium source code to determine (and document) how the multimedia internals worked.
\item Traced method calls on production devices to constructed a hierarchical representation of control flow through WebView's multimedia internals under Chromium's multi-threaded, asynchronous architecture.
\end{itemize}
>>>>>>> 54d1cc3b582b224540ca2287790f91484845008b
\end{itemize}}
%------------------------------------------------
\end{entrylist}

%----------------------------------------------------------------------------------------
%   PROJECTS SECTION
%----------------------------------------------------------------------------------------
\section{projects}

\begin{entrylist}
%------------------------------------------------
\entry
{Jan-June 2014 }
{CryptoBot}
{}
{\emph{Cryptocurrency trading platform and algorithmic trading bot engine} \\
Automated bots analyze the Bitcoin market to detect the presence of other trading strategies; this information can be used to exploit discovered trading strategies.
\begin{itemize}
\item I designed and implemented a distributed, load-balancing framework to run bots on our computing cluster using Apache Thrift.
\item I wrote a Django frontend, which leverages Cython to create a native Python interface to our C++ framework.
\item I act as sysadmin for our servers, and DBA for our NuoDB deployment, providing large numbers of bots with concurrent access to real-time trade data in a fault-tolerant manner.
\end{itemize}
}

\entry
{Aug 2013}
{BatStats}
{}
{\emph{Battery widget written in C, using Linux kernel headers}
\begin{itemize}
\item I use the Linux kernel's C API to query the device's power supply.
\item If discharging, the program extrapolates the estimated total battery life.
\item If charging, it displays the estimated time of charge completion, and the day of the week if that time crosses the date line.
\end{itemize}
}

\entry
{Sept 2014 -}
{Introsort}
{}
{\emph{Hybrid sorting algorithm}
\begin{itemize}
\item Quicksorts until a recursion depth of log(n), then heapsorts
\item Small chunks are passed to an assembly implementation of shellsort
\item Outperforms C++ std::sort and std::unstable\_sort for most test cases
\end{itemize}
}


\entry
{}
{Hackathons}
{}
{Google 24 Hours of Good - LA (November 2012), LA Hacks (April 2013), Hacktech (January 2014)}
%------------------------------------------------
\end{entrylist}

\newpage

%----------------------------------------------------------------------------------------
%	COURSES SECTION
%----------------------------------------------------------------------------------------

\section{courses}

\begin{entrylist}
%------------------------------------------------
\entry
{}
{California Institute of Technology}
{}
{\emph{Course listings:}
\begin{itemize}
\item \emph{CS 21}: Decidability and Tractability. Exploring the fundamental limits of (efficient) computation
\item \emph{CS 24}: Computer Systems. Hardware-software interface, computer architecture, and operating systems
\item \emph{CS 38}: Algorithms. Major algorithm design techniques and methods for identifying intractibility
\item \emph{CS 11}: Coding Project. Framework/middleware development for CryptoBot
\item \emph{CS 90}: Research. Developing algorithmic trading strategies in cryptocurrency markets like Bitcoin for use by CryptoBot, with Prof. Adam Wierman
\item \emph{CS/EE 145}: Networking Project. Deliverables include a functional CryptoBot project, and accompanying research paper, with Prof. Wierman
\item \emph{CS 115}: Functional Programming. Haskell, functional programming theory
\item \emph{CS 121}: Relational Databases. Extensive practical work with SQL
\item \emph{CS/EE 143}: Communication Networks. Mechanisms and protocols in communication networks, and mathematical models for their analysis
\item \emph{CS/CNS/EE 156A}: Learning Systems. Theory, algorithms, and applications of automated learning
\item \emph{CS 179}: GPU Programming. CUDA \& OpenGL visualization and simulation
\item \emph{ACM/EE/CMS 116}: Stochastic Processes \& Modeling. Fundamental ideas and techniques of stochastic analysis \& modeling
\item \emph{CS 11}: Computer language workshop. C, C++
\end{itemize}}
%------------------------------------------------
\end{entrylist}

%----------------------------------------------------------------------------------------
%	COMMUNICATION SKILLS SECTION
%----------------------------------------------------------------------------------------

\section{leadership \& communication skills}

\begin{entrylist}
%------------------------------------------------
\entry
{2008-2012}
{Speech \& Debate}
{Team Captain}
<<<<<<< HEAD
{I successfully competed in public speaking and debate events at league and charity tournaments at the state and national level.}
\entry
{Summer 2013}
{Lab126 "Brown Bag" Presentation}
{Information and QA session}
{I presented details of my work and relevant systems to co-workers \& managers over the course of an informal lunch session, followed by a period of questions.}
\entry
{Fall 2014}
{CS 143 Project}
{Project Manager}
{I led the design \& development of my group's network simulator, managed the division of development responsibilities, and organized \& led team meetings.}
%------------------------------------------------
\end{entrylist}

%----------------------------------------------------------------------------------------
%	INTERESTS SECTION
%----------------------------------------------------------------------------------------

\section{interests}

\textbf{professional:} leadership training, backend/framework development, distributed systems, machine learning, web development, automation, concurrent programming, parallel programming, OS/systems development, functional programming
\\
\textbf{personal:} long-distance running, cooking, Aikido, fencing, billiards, hiking, backpacking

\end{document}
